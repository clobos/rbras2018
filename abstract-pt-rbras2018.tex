\documentclass[12pt, a4paper]{article}

%-----------------------------------------------------------------------
% Preamble by the organize commitee

\usepackage[utf8]{inputenc}
\usepackage[brazil]{babel}
\usepackage[margin=2.5cm]{geometry}
\usepackage{setspace}
\usepackage{indentfirst}
\usepackage{graphicx}
\usepackage{xcolor}
\usepackage{fancyhdr}
\usepackage{url}
\usepackage{enumerate}
\usepackage{amsmath, amsthm, amsfonts, amssymb, amsxtra}
\usepackage{bm}

\pagestyle{fancy}
\fancyhf{}
\lhead{$63^{\textrm{a}}$ RBras}
\rhead{23 a 25 de maio de 2018, Curitiba - PR}
% \cfoot{\thepage}
\renewcommand{\headrulewidth}{0.4pt}
\addtolength{\headheight}{12.0pt}

\makeatletter
\def\@xfootnote[#1]{%
  \protected@xdef\@thefnmark{#1}%
  \@footnotemark\@footnotetext}
\makeatother

\usepackage{hyperref}
\definecolor{mycol}{rgb}{0.0, 0.0, 0.5}
\urlstyle{tt}
\makeatletter
\hypersetup{
  pdftitle={\@title},
  pdfauthor={\@author},
  colorlinks=true,
  linkcolor=mycol,
  citecolor=mycol,
  filecolor=mycol,
  urlcolor=mycol,
  bookmarksdepth=4
}
\makeatother

%-----------------------------------------------------------------------
% Init the document

\begin{document}
\onehalfspacing

%-------------------------------------------
% Título
\begin{center}
  \textbf{
    \Large{Modelos duplos COM-Poisson: modelando média e dispersão
      na análise de contagens}
  } \\[1em]
\end{center}

%-------------------------------------------
% Autores
\begin{flushright}
  {\bf Eduardo Elias Ribeiro Junior}
  \footnote[$\dagger$]{Contato:
    \href{mailto:jreduardo@usp.br}{\tt jreduardo@usp.br}}
  \footnote[1]{Departamento de Ciências Exatas (LCE) - ESALQ-USP}
  \footnote[2]{Laboratório de Estatística e Geoinformação (LEG) -
    UFPR}\\
  {\bf Clarice Garcia Borges Demétrio} \footnotemark[1]
\end{flushright}

\vspace*{0.5cm}

\noindent Para análise de dados em forma de contagens, comumente,
modela-se a média da variável resposta em termos de covariáveis. A
relação média--variância é determinada ao especificar a distribuição de
probabilidades das contagens. Dessa forma, a variância das contagens se
relaciona com as covariáveis apenas por meio de suas médias. Nesse
artigo, propõem-se os modelos duplos COM-Poisson, em que adota a
distribuição COM-Poisson para a média para as contagens e modela-se
ambos os parâmetros, de média e de dispersão, em termos de
covariáveis. Essa abordagem garante flexibilidade suficiente para
identificar covariáveis influentes tanto na média quanto na dispersão
das contagens em casos de sub, equi e superdispersão. Os modelos
propostos são ajustados pelo método da máxima verossimilhança e a
inferência sob os parâmetros é baseada na distribuição assintótica dos
estimadores de máxima verossimilhança. A metodologia é aplicada para
análise de um estudo em biometria. Os resultados dos modelos duplos
COM-Poisson apresentaram melhorias em termos de ajuste e interpretação
quando comparados aos resultados dos modelos COM-Poisson
convencionais. As implementações computacionais para ajuste dos modelos
duplos COM-Poisson são disponibilizadas em material
suplementar.\\

\noindent{\bf Palavras-chave}:
{\it Distribuição COM-Poisson, Modelagem da dispersão,
     Modelos duplos COM-Poisson}.\\

\end{document}